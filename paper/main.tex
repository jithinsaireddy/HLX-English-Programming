\documentclass[journal]{IEEEtran}

% *** PACKAGES ***
\usepackage{cite}
\usepackage{graphicx}
\usepackage{amsmath,amssymb}
\usepackage{listings}
\usepackage{url}
\usepackage[hidelinks]{hyperref}

\begin{document}

\title{HLX on NLVM: Deterministic Execution of Controlled English for IoT with RTOS and Web of Things Artifacts}

\author{First A. Author, Second B. Author, and Third C. Author%
\thanks{Manuscript submitted Month DD, 2025. Authors are with <Affiliation/Department/Address>. Corresponding author: <email>.}}

\maketitle

\begin{abstract}
We present \emph{HLX}, a controlled-English programming layer for Internet-of-Things (IoT) and cyber-physical systems that compiles human-readable rules into deployable artifacts: real-time operating system (RTOS) code (Rust/FreeRTOS or Zephyr C), edge deployment manifests, and W3C Web of Things (WoT) Thing Descriptions (TD). HLX is built on a general \emph{English Programming Language} (EPL) and executes deterministically by construction: timed predicates (e.g., \emph{``for 600\,ms''}), hysteresis, and cooldown are compiled to explicit schedules and state machines, yielding reproducible behavior and auditable logs. Across representative device policies—industrial boiler overpressure, HVAC thermostat hysteresis, multi-sensor imbalance, pipeline leak detection, and hospital CO$_2$ alerts—HLX demonstrates (i) deterministic rule execution with traceability to source lines, (ii) low jitter on RTOS targets, and (iii) standards-based interop via automatically generated WoT TDs. We release the specification, compiler, examples, and evaluation scripts.
\end{abstract}

\begin{IEEEkeywords}
Controlled Natural Language, Internet of Things, Real-Time Operating Systems, Web of Things, Programming Languages, Cyber-Physical Systems, Determinism.
\end{IEEEkeywords}

\section{Introduction}
Programming IoT/edge systems remains split between (1) end-user \emph{trigger–action} rules with limited timing semantics and (2) low-level firmware requiring specialized expertise. Natural language interfaces via LLMs can generate code, but introduce stochasticity and weak auditability. We ask: \emph{Can we execute precise English directly and deterministically for IoT control while emitting portable, standards-based artifacts?}

We answer with \textbf{HLX}, a controlled-English language for devices, sensors, actuators, timing windows, hysteresis, cooldown, and side effects (publish/store). HLX compiles to RTOS tasks/ISRs and generates WoT Thing Descriptions (TD 1.1) for interop. HLX builds atop our general \textbf{EPL}/NLVM foundation but is domain-focused on IoT. Unlike assistant languages~\cite{ThingTalk22,GeniePLDI19} or trigger–action platforms~\cite{UrIFTTT16,BrackenburyTAP19}, HLX provides deterministic execution and explicit timing on resource-constrained devices, and unlike CNLs used for logic/specification~\cite{KuhnCNL14,ACEManual}, HLX targets deployable embedded code and WoT descriptors~\cite{WoT11}.

\textbf{Contributions.} (1) HLX: a controlled-English language with real-time idioms (windows, hysteresis, cooldown) and fail-closed disambiguation; (2) a compiler that emits RTOS code (Rust/FreeRTOS, Zephyr C), edge manifests, and WoT TDs; (3) deterministic execution and traceability with source-line mapping; (4) an evaluation on realistic scenarios with latency/jitter, resource usage, and interop validation.

\section{Background and Related Work}
\subsection{Controlled Natural Languages (CNLs)}
CNLs restrict English to reduce ambiguity and enable machine processing~\cite{KuhnCNL14}. Attempto Controlled English (ACE) compiles specifications to formal logic~\cite{ACEManual}. HLX adopts the \emph{discipline} of CNL but targets executable device control with concrete artifacts, not only logical models.

\subsection{English-like and Assistant Languages}
Inform~7 uses natural-language-like syntax for interactive fiction; AppleScript targets desktop automation~\cite{InformSite,AppleScriptGuide}. Assistant formalisms such as ThingTalk/Genie~\cite{ThingTalk22,GeniePLDI19} represent user intents for virtual assistants and web APIs, rather than compiling to RTOS or WoT TD.

\subsection{Trigger–Action Programming (TAP)}
Large-scale studies of IFTTT/TAP highlight expressivity and reliability issues (duplication, ambiguity, unintended triggers)~\cite{UrIFTTT16,BrackenburyTAP19}. HLX addresses common failure modes via timed predicates, hysteresis/cooldown, and compile-time checks.

\subsection{Interoperability Standards and RTOS Targets}
The W3C WoT TD 1.1 standard defines a protocol-agnostic information model for device descriptions and interop~\cite{WoT11}. We automatically emit TDs and validate them. For deployment, we target Zephyr and FreeRTOS as representative RTOSes in constrained environments~\cite{ZephyrDocs,FreeRTOSDocs}.

\section{The HLX Language}
HLX is a controlled-English DSL with explicit devices, sensors, actuators, periods/units, timed conditions, hysteresis, cooldown, and side effects:
\begin{itemize}
  \item \textbf{Device model:} \texttt{Device "Boiler-A" at mqtt://plant/boilerA}
  \item \textbf{Sensors/Actuators:} \texttt{Sensor "pressure" unit kPa period 200 ms}; \texttt{Actuator "relief\_valve" actions open, close}
  \item \textbf{Timed predicates:} \texttt{If pressure > 180 kPa for 600 ms then ...}
  \item \textbf{Stability:} \texttt{with hysteresis 5 \% and cooldown 5000 ms}
  \item \textbf{Effects:} \texttt{open relief\_valve; publish event ...; store last 5000 ms ...}
\end{itemize}

\noindent\textbf{Example (Boiler—Overpressure).}
\begin{lstlisting}[language={},basicstyle=\ttfamily\small]
Device "Boiler-A" at mqtt://plant/boilerA
Sensor "pressure" unit kPa period 200 ms
Actuator "relief_valve" actions open, close

If pressure > 180 kPa for 600 ms then
  open relief_valve
  publish event "overpressure" with timestamp and value
  store last 5000 ms of pressure to table "incidents"
\end{lstlisting}

\noindent \textbf{Semantics.} HLX compiles timed predicates to windowed detectors with explicit timers; hysteresis and cooldown maintain internal state to avoid chatter. Ambiguity (unknown device/action) is a compile error (fail-closed). Units and time dimensions are checked statically.

\section{Compiler and Artifacts}
\textbf{Lowering to RTOS.} HLX rules compile to tasks/ISRs, timers, and queues. For FreeRTOS we generate Rust scaffolding; for Zephyr we generate C (main entry, k\_timer callbacks). The same HLX source yields:
\begin{enumerate}
  \item RTOS code (\texttt{rtos.rs} or \texttt{zephyr\_main.c})
  \item Edge/gateway manifest (e.g., MQTT topics, deployment metadata)
  \item WoT TD 1.1 JSON-LD (\texttt{thing\_description.json})
\end{enumerate}

\noindent \textbf{Observability.} The compiler attaches source spans to generated actions; runtime logs include rule IDs, window start/stop, hysteresis thresholds, and cooldown state for audit.

\section{Implementation}
We implement a single-pass parser for HLX with typed AST, a checker for units/time, and backends for RTOS, manifests, and TD. The web demo provides “Compile” (generate artifacts) and “Run Demo” (simulate sensor streams, print logs). Artifacts are downloadable and shareable for review.

\section{Evaluation}
We evaluate determinism, latency, and interop across six scenarios:

\noindent\textbf{Scenarios.} Boiler–Overpressure; HVAC–Thermostat (hysteresis/cooldown); Water Tank–Imbalance (multi-sensor); Pipeline–Leak (differential pressure); Hospital–CO$_2$ (fast/slow); Cold-chain–Freezer (fast/slow).

\noindent\textbf{Metrics.} (1) run-to-run equivalence (identical traces); (2) decision latency/jitter (p50/p99); (3) CPU/RAM footprint; (4) number of prevented oscillations (hysteresis); (5) WoT TD validation success.

\noindent\textbf{Baselines.} We compare to representative TAP rules and an assistant formalism (where applicable) to illustrate missing timing semantics and interop gaps.

\noindent\textbf{Result Summary.} HLX executes deterministically with low jitter on RTOS targets, prevents repeated/oscillatory actions under hysteresis/cooldown, and produces WoT TDs that validate against the 1.1 specification.

\section{Discussion}
\textbf{When HLX is preferable.} Time-bound device policies with stability requirements, auditable behavior, and standards-based interop. \textbf{Limitations.} Complex multi-device coordination and rich computations are intentionally outside the core HLX (can be offloaded to services). \textbf{EPL/NLVM foundation.} Our general EPL compiles to a typed IR and NLVM for host execution; HLX leverages EPL’s parsing/checking strategies but targets device deployment.

\section{Design Choice: Is a Bytecode Backend Required for HLX?}
Not for this paper’s goals. HLX’s RTOS codegen and WoT TD artifacts suffice for deterministic execution and interop. A future HLX$\rightarrow$bytecode (HLX$\rightarrow$NLBC) backend with a verifier could further improve portability and sandboxing (analogous to eBPF/Wasm ecosystems) but is optional for present contributions.

\section{Threats to Validity}
Hardware diversity, RTOS configuration, network variance, and authoring bias may affect results. We mitigate via scripted workloads, two RTOS targets, and public artifacts.

\section{Related Work (Consolidated)}
CNLs~\cite{KuhnCNL14,ACEManual}; English-like/assistant languages~\cite{InformSite,AppleScriptGuide,ThingTalk22,GeniePLDI19}; TAP studies~\cite{UrIFTTT16,BrackenburyTAP19}; WoT~\cite{WoT11}; RTOS docs~\cite{ZephyrDocs,FreeRTOSDocs}.

\section{Conclusion}
HLX demonstrates that controlled English can be compiled into deterministic, auditable, and interoperable IoT programs. By emitting RTOS code and WoT TDs from a single source, HLX bridges human-readable policy authoring and deployable device software. We release the language, compiler, examples, and evaluation scripts to foster reproducible research.

\section*{Acknowledgment}
We thank the open-source communities behind Zephyr, FreeRTOS, and W3C WoT.

\bibliographystyle{IEEEtran}
\bibliography{refs}

\end{document}
